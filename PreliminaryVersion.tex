\documentclass[12pt]{article}
\usepackage[utf8]{inputenc}
\usepackage{geometry}
\usepackage{booktabs}
\usepackage{array}
\usepackage{hyperref}

\geometry{margin=1in}
\title{Transit Route Planner (Open, Privacy-first, Offline-capable)}
\author{Team Project}
\date{\today}


\begin{document}
\maketitle

\begin{abstract}
This preliminary document describes an open-source, privacy-respecting transit route-planning application with extended platform support (Android, iOS, mobile Linux distributions such as postmarketOS, and feature-phone platforms such as KaiOS). The application focuses on highly configurable multi-criteria route planning, offline operation for core features, and community-driven development and verification.
\end{abstract}

\section{Objectives}
The project aims to design, implement and demonstrate a practical, open-source public-transit route planner that:
\begin{itemize}
  \item supports fine-grained multi-criteria route selection (line, operator, vehicle type, transfers, payment methods, time/distance priorities, etc.);
  \item provides limited but useful offline functionality (maps, route planning, and navigation when pre-downloaded);
  \item minimizes data collection and avoids data gathering for sale; and
  \item supports broad platforms (Android, iOS, postmarketOS, KaiOS where feasible).
\end{itemize}

\section{Indicators and Measurable Goals}
\subsection{Key Performance Indicators (KPIs) and justification}

\begin{itemize}
  \item \textbf{Correctness of route generation:} Achieve at least \textbf{95\%} agreement with official timetable-based routes in selected pilot regions (measured versus transit agency schedules).\\
  \emph{Reason:} High correctness ensures users can rely on recommended itineraries; 95\% is realistic when using authoritative schedule feeds (GTFS) and careful parsing/validation (GTFS is widely used by transit agencies).

  \item \textbf{Offline functionality coverage:} Provide offline map tiles and routing data covering \textbf{90\%} of the pilot-region road and transit network within pre-download size limits (e.g., <500 MB).\\
  \emph{Reason:} Offline support must be practically useful without excessive storage; 90\% coverage balances usability and size constraints, relying on tile compression and vector-based route graphs (Map SDKs support offline packaging)~\cite{mapbox_offline}.

  \item \textbf{Privacy compliance:} The application must not transmit or log personal location traces to third-party servers by default; optional telemetry must be opt-in and fully documented.\\
  \emph{Reason:} Clear privacy design differentiates the project from commercial apps that collect user data (see Google policies and recent enforcement actions)~\cite{google_privacy,google_settlement}.

  \item \textbf{Platform support:} Release functional clients for at least \textbf{Android, iOS,} and \textbf{one mobile Linux distribution} (postmarketOS) within the project timeframe.\\
  \emph{Reason:} Supporting a Linux mobile distribution demonstrates broader accessibility; postmarketOS has active development and received community funding (NLnet / NGI funding)~\cite{postmarketos,nlnet_postmarketos}.

  \item \textbf{Open-source deliverables:} Publish at least \textbf{3 reusable components} (map rendering module, offline packager, and multi-criteria routing engine) under an OSI-compatible license (e.g., MIT).\\
  \emph{Reason:} Delivering reusable modules accelerates uptake and community contributions, which is a core declared goal of the project.

  \item \textbf{User satisfaction in pilots:} Achieve a pilot user satisfaction score of at least \textbf{80\%} on usability and accuracy metrics (measured via surveys).\\
  \emph{Reason:} 80\% is a realistic target for early demos and indicates acceptable user experience for further development.
\end{itemize}

\section{State of the Art and Comparative Analysis}
\subsection{Short review}
Existing commercial services (Google Maps, Apple Maps) provide broad coverage and integrated traffic/vehicle-position feeds, but typically collect user data and limit customization. OpenStreetMap (OSM) provides community-driven map data with variable completeness across regions (OpenStreetMap Wiki)~\cite{osm_completeness}. Map SDKs such as Mapbox offer offline packaging that can be used to support offline operation in apps~\cite{mapbox_offline}. Mobile Linux distributions (postmarketOS) and feature-phone platforms (KaiOS) expand potential device coverage beyond Android/iOS~\cite{postmarketos,kaios_home}.


\subsection{Comparison table: Google Maps vs Open-Source Approaches}

Relevant documentation and discussions on offline support and privacy are available in Mapbox's offline docs and Google privacy resources~\cite{mapbox_offline,google_privacy}.

\begin{table}[h!]
\centering
\caption{Comparison of mapping/navigation approaches relevant to this project}
\begin{tabular}{|p{4cm}|p{5cm}|p{5cm}|}
\hline
\textbf{Feature} & \textbf{Commercial (Google Maps)} & \textbf{Open-source (OSM + SDKs)} \\ \hline
License / cost & Proprietary, API fees for some usage & Open data; SDK costs may apply for hosted services \\ \hline
Data control & Centralized, vendor-controlled & Community editable (OSM), full data export \\ \hline
Offline support & Limited, vendor-dependent & Strong support via SDKs and offline tile packs \\ \hline
Privacy & Extensive telemetry/collection (subject to policies) & Can be designed to minimise collection; transparent codebase \\ \hline
Customization & Limited by API & High: data and rendering fully customizable \\ \hline
Platform reach & Excellent on Android/iOS & High when targeting common SDKs; additional work needed for postmarketOS/KaiOS \\ \hline
\end{tabular}
\label{tab:comparison}
\end{table}


\section{Impact and Risk Assessment}

\subsection{Impact}
The project delivers technological, social and educational value:
\begin{itemize}
  \item \textbf{Technological:} Practical open-source route planner with offline capabilities and multi-criteria filtering usable by NGOs, small municipalities, and developers.
  \item \textbf{Social / Environmental:} Better public-transit usability can increase public transport uptake and reduce car usage, contributing to emissions reductions.
  \item \textbf{Economic:} Reduces lock-in to proprietary services and provides reusable components for local tech ecosystems.
  \item \textbf{Educational:} Encourages contributions from students and junior developers, building skills in geospatial software and open data practices.
\end{itemize}

\subsection{Risk Analysis}
\begin{table}[h!]
\centering
\caption{Key Risks and short description}
\begin{tabular}{|p{4cm}|p{10cm}|}
\hline
\textbf{Risk Type} & \textbf{Description} \\ \hline
Technical & Integration of heterogeneous transit data (GTFS variants, regional formats) and offline packaging constraints may complicate development. \\ \hline
Data availability & Some regions lack complete open transit data or OSM coverage; data freshness varies~\cite{osm_completeness}. \\ \hline
Organisational & Maintaining coordination in an open-source project and handling contributions can introduce overhead. \\ \hline
Financial & Hosting, map tiles, and continuous maintenance require funding beyond initial development. \\ \hline
User adoption & Competing with established products requires strong user-focused differentiators (privacy, customization, offline support). \\ \hline
Team experience & The team is primarily composed of young developers with limited production experience; this may slow early development and increase the need for mentorship and external collaboration. \\ \hline
\end{tabular}
\label{tab:risks}
\end{table}

\section{Planned Outputs and Timeline (high level)}
\begin{itemize}
  \item \textbf{M1 (0--3 months):} Requirements, data source inventory (GTFS feeds, OSM), technical architecture, and initial prototypes for map rendering.
  \item \textbf{M2 (3--9 months):} Offline packaging tool, basic routing with multi-criteria filtering, Android and postmarketOS proof-of-concept clients.
  \item \textbf{M3 (9--18 months):} iOS client, broader pilot deployment, documentation, and publication of 3 reusable components under MIT license.
\end{itemize}

\section{Deliverables}
\begin{enumerate}
  \item Preliminary report and data inventory (LaTeX + PDF).
  \item Source code repository with MIT license and three modular components.
  \item Pilot dataset and offline package for one test region.
  \item User evaluation report (pilot surveys and metrics).
\end{enumerate}

\bibliographystyle{plain}
\bibliography{PreliminaryVersion_Refs}

\end{document}
