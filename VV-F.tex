
\documentclass[12pt,a4paper]{article}
\usepackage[utf8]{inputenc}
\usepackage{geometry}
\geometry{margin=2.5cm}
\usepackage{graphicx}
\usepackage{hyperref}
\usepackage{longtable}
\usepackage{booktabs}
\usepackage{caption}

\title{VV-F: Material Intent of the Project \\ Open-Source, Privacy-First Transit Route Planner with Offline Support}
\author{Student Team -- 1st Year University Project, FIIT STU}
\date{\today}

\begin{document}
\maketitle
\tableofcontents
\newpage

\section{Excellence}

\subsection{1.1 Project Objectives and Innovation}
This project proposes the creation of an open-source, privacy-first public transit route planner that functions both online and offline, with extended platform support (Android, iOS, and mobile Linux distributions such as postmarketOS). The system will focus on ethical data handling, transparency, and accessibility, distinguishing it from commercial solutions like Google Maps that depend on centralized data collection and user tracking~\cite{google_privacy}.

The project allows users to plan routes based on multiple parameters — type of vehicle, number of transfers, preferred transit company, estimated duration, and route length — while maintaining offline capability for all features that do not require real-time updates. Its open-source nature encourages reuse, transparency, and community contributions, enabling independent verification that no user data is collected or monetized~\cite{opensource_values}

\textbf{Key measurable objectives:}
\begin{itemize}
  \item Develop a working prototype for at least one mobile platform.
  \item Achieve at least 85\% routing accuracy when compared to existing online tools, such as Google Maps or Mapy.cz.
  \item Provide full offline map and route planning for at least one pilot area.
  \item Release the full source code under an open-source license.
  \item Create basic project documentation to enable others to test, use, or improve the application.
\end{itemize}

Although the team lacks prior experience, the project is designed as a learning-oriented initiative. Its purpose is to develop technical, research, and teamwork skills while contributing a small but functional open-source mobility solution~\cite{opensource_learning}. 

\subsection{1.2 Methodology}
The development process follows an iterative and incremental approach. The team will begin by studying open data standards such as GTFS (General Transit Feed Specification)~\cite{gtfs_spec} and OpenStreetMap (OSM)~\cite{osm_wiki}. Early milestones focus on understanding the data formats, developing simple import and routing functions, and gradually extending features to include user preferences and offline storage.

Testing and evaluation will rely on comparing generated routes with official public transport timetables and known commercial apps. Progress will be tracked through Git version control, ensuring transparency and documentation of development steps~\cite{osm_completeness}.

\subsection{1.3 Research Team and Learning Goals}
The project is conducted by a small group of four first-year university students. None of the team members have prior experience with large-scale software or applied research, making this project primarily an educational and exploratory effort. Each student will be responsible for one key domain — such as data processing, mobile interface, system architecture, or documentation.

While there is no formal Principal Investigator in the traditional research sense, the group operates collaboratively, with each member contributing equally. The work is guided by academic supervisors who provide methodological advice but allow full independence in implementation. The project’s main strength lies in the students’ motivation to learn, collaborate, and contribute meaningfully to open-source development~\cite{nlnet_case_study}. 

\section{Impact}

\subsection{2.1 Expected Impact and Benefits}
Despite its modest scale, the project can generate measurable academic and social value. Its open-source nature ensures transparency and educational reuse by other students or developers. The resulting application prototype will demonstrate that even beginner teams can contribute to practical, privacy-respecting software~\cite{foss_impact}. 

\textbf{Expected impacts:}
\begin{itemize}
  \item \textbf{Educational:} The team members gain practical experience with applied research and software development tools.
  \item \textbf{Scientific:} The project documents a case study in open-source, non-commercial software design for mobility planning.
  \item \textbf{Social:} Provides a transparent, data-respecting alternative to proprietary routing systems~\cite{gdpr_framework}.
  \item \textbf{Technical:} Produces reusable open-source components that others may extend or integrate into their own projects.
\end{itemize}

The project also contributes to awareness about data privacy in public digital services and encourages students to adopt ethical principles in technology design.

\subsection{2.2 Dissemination of Results}
All project outputs — source code, documentation, and datasets — will be published publicly in a GitHub repository under an open license. A small project website will host the compiled app and explain the goals, methodology, and ethical principles. Dissemination will occur mainly within the university environment and through open-source communities~\cite{opensource_dissemination}. 

If successful, the team plans to present the project at student technology fairs or conferences focused on open software and digital privacy.

\section{Implementation}

\subsection{3.1 Work Plan and Milestones}
The project is planned for a total duration of 12 months and divided into the following work packages (WPs):

\begin{longtable}{p{1.5cm}p{4cm}p{7.5cm}p{2.5cm}}
\toprule
\textbf{WP} & \textbf{Title} & \textbf{Main tasks and deliverables} & \textbf{Duration} \\
\midrule
WP1 & Research and data preparation & Study GTFS and OSM standards; gather test datasets; prepare basic data import scripts & Months 1--3 \\
WP2 & Routing prototype & Implement basic routing algorithm using open data; verify route correctness & Months 4--6 \\
WP3 & User interface and offline support & Develop a simple user interface and test offline operation on at least one platform & Months 7--9 \\
WP4 & Evaluation and documentation & Compare prototype with commercial apps; write documentation; publish results & Months 10--12 \\
\bottomrule
\end{longtable}

\subsection{3.2 Project Management and Risks}
The team will manage tasks collaboratively using online tools such as Git and shared documentation platforms. Decisions are made collectively, promoting equal participation. Academic mentors may provide limited supervision and consultation.

\textbf{Identified risks:}
\begin{longtable}{p{4.5cm}p{9cm}}
\toprule
\textbf{Risk} & \textbf{Description and Mitigation} \\
\midrule
Lack of experience & The team has no prior project experience, which may slow progress. Mitigation: learning from open-source examples, tutorials, and mentor advice. \\
Time management & Balancing studies and project work may affect delivery. Mitigation: weekly planning and task division. \\
Technical complexity & Some features (e.g., offline routing) may exceed skill level. Mitigation: start simple and focus on core functionality first. \\
Data quality & Open datasets may contain errors or gaps. Mitigation: focus testing on small, well-documented areas. \\
\bottomrule
\end{longtable}

\subsection{3.3 Budget and Resources}
The project requires minimal funding — mostly for basic hardware (e.g., shared testing devices), internet hosting for code repositories, and potential small-scale dissemination (e.g., university presentations). No external commercial software licenses are needed~\cite{opensource_budget}. 
\subsection{3.4 Institutional Support}
The project will be conducted under the supervision of the university’s department of applied informatics, which provides computing infrastructure, consultation, and academic guidance. The environment encourages student-led initiatives, open collaboration, and independent research.

\newpage
\bibliographystyle{plain}
\bibliography{PreliminaryVersion_Refs,PreliminaryVersion_Refs_Expanded}

\end{document}
