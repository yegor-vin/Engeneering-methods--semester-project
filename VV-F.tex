\documentclass[12pt,a4paper]{article}
\usepackage[utf8]{inputenc}
\usepackage{geometry}
\geometry{margin=2.5cm}
\usepackage{graphicx}
\usepackage{hyperref}
\usepackage{longtable}
\usepackage{booktabs}
\usepackage{caption}
\usepackage{array} 
\usepackage{url}   

\title{VV-F: Material Intent of the Project \\ Open-Source, Privacy-First Transit Route Planner with Offline Support}
\author{Student Team -- 1st Year University Project, FIIT STU}
\date{\today}

\begin{document}
\maketitle
\tableofcontents
\newpage

\section{Excellence}

\subsection{Project Objectives, Intent, and Innovation}
The intent of the project is to design and implement a privacy-first, open-source public transit route planner operating both online and offline, with support for Android, iOS, and mobile Linux platforms. The solution addresses growing privacy, societal and academic concerns regarding data collection, algorithmic opacity, and dependency on commercial routing services~\cite{google_privacy}. By enabling routing without user tracking, the project promotes ethical, transparent, and trustworthy digital mobility tools.

The originality of the project lies in combining:
\begin{itemize}
    \item complete offline routing based on open datasets (GTFS and OSM).
    \item multi-criteria routing customizable by user preferences.
    \item full transparency of algorithms and data transformations.
    \item cross-platform deployment.
\end{itemize}
This contributes to the scientific and technological domain of intelligent transport systems, privacy-preserving computation, and open-source mobility applications.

\textbf{Application level of results:}  
The expected output is a functioning prototype capable of planning public transport routes for at least one region. The prototype will be made openly available and reproducible, allowing further development into a fully deployable mobility tool by future student or research teams.

\textbf{Main project objectives:}
\begin{itemize}
    \item develop a fully functional routing prototype using GTFS and OSM data.
    \item achieve at least 80\% correctness compared with commercial tools.
    \item support offline storage and routing in a selected area.
    \item release the complete source code, documentation, and datasets under an open license.
    \item demonstrate a transparent, privacy-respecting alternative to proprietary routing systems.
\end{itemize}

These objectives are realistic considering the project's scope and the availability of open-source software ecosystems, existing routing methodologies, and academic supervision.

\subsection{State of Knowledge and Scientific Foundations}
The project builds on established open data standards used internationally:
\begin{itemize}
    \item \textbf{GTFS (General Transit Feed Specification)} for transit schedules and routes~\cite{gtfs_spec},
    \item \textbf{OpenStreetMap (OSM)} for geographic and topological information~\cite{osm_wiki, osm_completeness}.
\end{itemize}

In routing research, approaches such as the station-graph model and multi-criteria optimization have demonstrated high efficiency for public-transport navigation~\cite{delling_transit_routing}. These methods provide the theoretical foundation for the project’s routing implementation.

Current commercial systems like Google Maps provide highly accurate routing but rely on centralized infrastructure, user profiling, and opaque algorithms~\cite{google_privacy}. Recent open-source initiatives, including NLnet-funded mobility tools, show that independent, privacy-preserving routing is feasible but still underdeveloped for mobile devices~\cite{nlnet_case_study}. This identifies a clear research and innovation gap the project seeks to address.

\textbf{Feasibility:}  
All selected technologies—GTFS, OSM, open-source routing libraries, and cross-platform mobile frameworks—are mature, well-documented, and freely available. This ensures that the project goals are achievable within a student research environment.

\subsection{Methodology and Justification}
The project follows an iterative, incremental development process. Each iteration introduces small, testable improvements based on feedback and evaluation results.

\textbf{Methodological components:}
\begin{itemize}
    \item \textbf{Data analysis and preparation:} validating, and normalizing GTFS and OSM data for offline use.
    \item \textbf{Algorithmic development:} implementing a routing engine inspired by multi-criteria public-transport routing approaches.
    \item \textbf{Software prototyping:} building a mobile application using open-source frameworks to ensure reproducibility.
    \item \textbf{Evaluation:} validation against commercial apps and official timetables.
    \item \textbf{Documentation and dissemination:} maintaining a Git repository, user guide, and technical documentation.
\end{itemize}

\textbf{Justification:}  
This methodology ensures continuous progress and aligns with best practices for small-scale applied research projects. Iterative development allows early detection of errors, while open-source publication guarantees transparency and reproducibility.

\subsection{Excellence of Supervisors and Institutional Competence}
Although the student team has no formal Principal Investigator, the project is carried out under the academic supervision of experienced faculty members at FIIT STU. The faculty provides:
\begin{itemize}
    \item expertise in software engineering, routing algorithms, and open-source development,
    \item access to institutional infrastructure, version control, consultation hours, and laboratories,
    \item methodological guidance necessary for completing an applied research assignment.
\end{itemize}

FIIT STU has a strong reputation in computer science education and research, especially in informatics, data processing, and mobile application development. The institution regularly supports student-led research projects and contributes to national and international initiatives in software innovation.

\subsection{Research Team Expertise and Development Vision}
The team consists of four first-year university students motivated to learn and participate in applied research. Despite limited prior experience, their competencies cover:
\begin{itemize}
    \item software development and programming fundamentals,
    \item open-source community practices,
    \item mobile application design,
    \item documentation and project management.
\end{itemize}

The project will strengthen their skills in:
\begin{itemize}
    \item data engineering,
    \item algorithmic design,
    \item version control and reproducible research,
    \item collaborative development workflows,
    \item academic writing and presentation.
\end{itemize}

\textbf{Vision for research development:}  
The project serves as the foundation for more advanced mobility research—for example, real-time routing, multi-modal transport prediction, or privacy-preserving location services. By completing the project, the team gains the capability to participate in more complex research initiatives in later years of study.

\subsection{Involvement of Young Researchers}
All core contributors are young researchers under 35 years of age. The project promotes early-stage involvement in applied research through:
\begin{itemize}
    \item hands-on experience with standard research methods,
    \item direct exposure to open-source technologies,
    \item mentorship from senior academics,
    \item teamwork experience relevant for future innovation projects.
\end{itemize}

The project supports the strategic goal of developing young research talent at FIIT STU and preparing students for future participation in national or European research programmes.


\section{Impact}

\subsection{Expected Impact and Benefits}
The proposed project delivers benefits in several areas of knowledge development, applied research, and social value. Although it is implemented by a student team, the project directly contributes to the advancement of privacy-preserving mobility technologies and demonstrates the feasibility of building routing tools without reliance on commercial tracking-based platforms~\cite{foss_impact}.

\textbf{Benefits for knowledge, applied research, and innovation:}
\begin{itemize}
    \item \textbf{New technological procedures:} The project develops a lightweight, privacy-first transit routing pipeline combining GTFS and OSM data, optimized for offline use. This contributes to research on low-resource and decentralized mobility tools.
    \item \textbf{Improved services:} It provides a transparent alternative to corporate routing systems that rely on user profiling. The application shows that routing services can function without storing personal data.
    \item \textbf{Social innovation:} By emphasizing privacy, openness, and accessibility, the project raises awareness about ethical technology design and encourages alternatives to data-exploiting services~\cite{gdpr_framework}.
    \item \textbf{Educational innovation:} It serves as a structured learning case for first-year students and may become a reusable educational example for future university courses on software engineering, open data, and applied research.
\end{itemize}

\textbf{Usability of results in Slovakia and abroad:}
\begin{itemize}
    \item \textbf{Local usability:}  
    The prototype can be directly tested with transit datasets from Slovak cities or regions, enabling students, researchers, or public transport enthusiasts to plan routes offline.
    \item \textbf{International usability:}  
    Because the system relies exclusively on global open data standards (GTFS and OSM), it can be adapted to any country without modification. Developers abroad can reuse code, contribute improvements, or deploy the tool in new regions.
    \item \textbf{Institutional usability:}  
    Universities, NGOs, and open-data groups may use the application as a demonstrator of privacy-first mobility solutions or as a teaching resource for courses involving routing algorithms and map data processing.
\end{itemize}

\textbf{Economic and societal benefits:}
\begin{itemize}
    \item \textbf{Saving resources:}  
    Offline functionality reduces mobile data consumption for users and minimizes server hosting costs, making the solution economically sustainable.
    \item \textbf{Improvement of human resources:}  
    The project strengthens students’ competencies in software engineering, data processing, documentation, and open-source collaboration, increasing their employability.
    \item \textbf{Indirect employment impact:}  
    Through open-source dissemination, the project may be used by startups, civic initiatives, and developers interested in mobility applications, lowering entry barriers for future innovation.
    \item \textbf{Quality of life and environment:}  
    Encouraging the use of public transport — through easier access to routing information — supports sustainable mobility, reduces emissions, and contributes to improved urban living conditions.
\end{itemize}

\subsection{Dissemination of Results}
To maximize the impact of the project, a clear dissemination strategy will be implemented that ensures transparency, reusability, and long-term accessibility of all outputs.

\textbf{Measures to maximize results:}
\begin{itemize}
    \item \textbf{Open-source publication:}  
    All code, documentation, and datasets will be publicly available on GitHub under a permissive license, ensuring that others may reuse or build upon the work~\cite{opensource_dissemination}.
    \item \textbf{Detailed documentation:}  
    The repository will include installation guides, API documentation, dataset preparation instructions, and examples to enable full reproducibility.
    \item \textbf{Engagement with the open-source community:}  
    The team will share progress updates on relevant platforms (e.g., Reddit’s r/opensource, OSM community forums) to attract contributors and testers.
    \item \textbf{Compatibility with standardized formats:}  
    Supporting GTFS and OSM maximizes interoperability, ensuring that results are usable by researchers, transit data curators, and developers worldwide.
\end{itemize}

\textbf{Communication of outputs:}
\begin{itemize}
    \item \textbf{University dissemination:}  
    Project results will be presented to faculty and classmates, with the potential for demonstration during student conferences, technology fairs, or project exhibitions.
    \item \textbf{Public website:}  
    A simple project webpage will host the compiled application, describe project goals, and provide instructions for reproducing or extending the tool.
    \item \textbf{Academic presentation:}  
    If the prototype reaches sufficient maturity, the team will prepare a short paper or poster summarizing the methodology and results, suitable for student conferences related to open-source technologies or digital privacy.
    \item \textbf{Community interactions:}  
    Blog posts, public demos, and recorded presentations will be used to explain how privacy-first routing works and why open-source alternatives matter.
\end{itemize}

\newpage
\section{Implementation}
\subsection{Work Plan and Milestones}

\begin{longtable}{p{1.4cm} p{4.3cm} p{7.3cm} p{2.2cm}}
\toprule
\textbf{WP} & \textbf{Title} & \textbf{Tasks and Deliverables} & \textbf{Duration} \\
\midrule
WP1 & Research and data preparation & Study GTFS and OSM standards; gather datasets; prepare import scripts & Months 1--6 \\
WP2 & Routing prototype & Implement routing algorithm using open data; verify correctness & Months 7--14 \\
WP3 & User interface and offline support & Develop UI; implement offline operation for one region & Months 15--22 \\
WP4 & Evaluation and documentation & Compare prototype with commercial apps; documentation; publish results & Months 23--24 \\
\bottomrule
\end{longtable}

\subsection{Project Management and Risks}
The team will manage tasks collaboratively using Git and shared documentation platforms. Decisions are made collectively, promoting equal participation. Academic mentors may provide limited supervision.

\textbf{Identified risks:}

\begin{longtable}{p{4.5cm} p{9cm}}
\toprule
\textbf{Risk} & \textbf{Description and Mitigation} \\
\midrule
Lack of experience & May slow progress. Mitigation: learn from open-source examples, tutorials, mentor feedback. \\
Time management & Balancing university workload may reduce available time. Mitigation: weekly task planning. \\
Technical complexity & Features like offline routing are difficult. Mitigation: start simple and scale up gradually. \\
Data quality issues & Open datasets contain inconsistencies. Mitigation: choose a small pilot region first. \\
\bottomrule
\end{longtable}

\subsection{Budget \& Resources}
Although the project is primarily software-based and developed by students, a defined budget is necessary to ensure functional development, testing, and dissemination. The project does not rely on any commercial software; all tools and libraries will be open-source~\cite{opensource_budget}. The budget covers only essential hardware, testing equipment, and minimal operational costs.
\textbf{Estimated total budget: \textbf{€480}}.
\textbf{Breakdown of required resources:}
\begin{itemize}
    \item \textbf{Testing devices (€250):}  
    At least one Android phone and one mobile Linux–capable device (e.g., PinePhone) are required to validate cross-platform functionality. The team already owns several personal smartphones, but a dedicated testing device ensures consistent testing and avoids data privacy issues.  
    \begin{itemize}
        \item Used Android device for debugging — \textit{€120}  
        \item PinePhone or similar Linux-based device — \textit{€130}
    \end{itemize}
    \item \textbf{Cloud hosting and repository costs (€0–€50):}  
    Although GitHub provides free hosting for open-source projects, optional expenses may arise for project website hosting or domain registration.  
    \begin{itemize}
        \item Optional domain name (1 year) — \textit{€12}  
        \item Optional hosting or VPS for demo builds — \textit{up to €38}  
    \end{itemize}
    \item \textbf{Power, storage, and development environment (€40):}  
    Includes external storage (USB drives or SD cards) for offline map datasets, backups, and device flashing.  
    \begin{itemize}
        \item 128GB SD card for map data — \textit{€20}  
        \item USB drive for backups and deployment — \textit{€15}  
        \item Miscellaneous small accessories — \textit{€5}
    \end{itemize}
    \item \textbf{Printing and dissemination (€100):}  
    Costs for producing printed posters, project documentation, and materials required for student presentations, conferences, or university showcases.  
    \begin{itemize}
        \item Poster printing (A1/A2) — \textit{€40}  
        \item Bound project documentation copies — \textit{€30}  
        \item Presentation materials and hand-outs — \textit{€30}
    \end{itemize}

    \item \textbf{Contingency reserve (€40):}  
    A small buffer for unforeseen expenses such as replacement chargers, cables, adapters, or other minor needs during development.
\end{itemize}

\subsection{Institutional Support}
The project is supported by FIIT STU, which provides access to infrastructure, consultation, and academic supervision.

\newpage
\bibliographystyle{plain}
\bibliography{PreliminaryVersion_Refs}

\end{document}
