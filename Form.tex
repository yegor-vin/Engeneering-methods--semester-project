\documentclass[12pt,a4paper]{article}
\usepackage[utf8]{inputenc}
\usepackage{geometry}
\geometry{margin=2.5cm}
\usepackage{graphicx}
\usepackage{hyperref}
\usepackage{longtable}
\usepackage{booktabs}
\usepackage{tabularx}
\usepackage{float}
\usepackage{microtype}
\sloppy
\usepackage[english,slovak]{babel}

\title{Forms VV-A to VV-F: Preliminary Version of Applied Research Project}
\author{Public Transport Development Team}
\date{\today}

\begin{document}
\maketitle


\section*{Form VV-A: Request for Applied Research}
\section*{Form VV-A1}

\textbf{01.Project Title:} Open-Source, Privacy-First Transit Route Planner with Offline Support\\
\textbf{02.Applicant:} \textit{FIIT STU}\\
\textbf{03.Project Type:} Applied Research\\
\textbf{04.Project acronym:} OOPTP\\
\textbf{05.R and D specialization:} To create unique map web-site and application combining existing features with improvements.\\
\textbf{06.R and D characterization:} Main purpose is to create a web-site and app without the weak sides of current solutions. It will be open-source, support offline maps and schedules, granular route preferences, OpenStreetMaps integration, user editing, cross-platform, and SMS ticketing.\\
\textbf{07.Project start:} 1.01.2026\\
\textbf{08.Project end:} 1.01.2028\\

\textbf{09.Annotation:} 
The project is aimed at creating an open-source and privacy-conscious open-source public transport planner, which is fully offline. The primary purpose here is to provide people with a dependable method of planning trips using buses, trains, or trams without having to use an internet connection, advertisements, or tracked data. Most of the available applications need internet connectivity at all times, gathered information about individuals, or only work in a small number of operating systems. It attempts to resolve these issues by prioritizing the users.

The application will have downloadable maps and timetable databases of various locations or business firms. In this manner, the users have the freedom of determining specifically what they wish to store within their gadget and only update those portions when their schedules are altered. When linked together, the app will be able to display live updates such as delays, route diversion, or replacement services. Users can also have more specific preferences e.g. what transport companies they want to choose, how many transfers they want to make, how far they are willing to walk or what places they would prefer to pass through.

Website and app will also be provided with a bare bone of editing as the user can correct mistakes or even add missing information in his/her own version, and submit his/her changes to be reviewed. The community-based approach can assist in maintaining the information unbiased without having to rely on official sources alone. The other feature that will be implemented is SMS ticket functionality, where the app will be able to prepare and send a text message to purchase a ticket, even in cases of offline mode.

To enable easy accessibility to all people, the project will support a broad platform based on Android, iOS, as well as Linux-based systems. Open data, offline availability and privacy protection: Bundle them all together and ??? will become a powerful, publicly accessible alternative to commercial transit apps. It tries to demonstrate that the means of the public transport can become open, secure and independent, designed to the community and by the community.\\

\textbf{10.Co-ordinating organization:} FIIT STU\\
\textbf{11.Required budget from the agency:} €300{,}000\\
\textbf{12.Financing from other sources:} None\\
\textbf{13.Total project budget:} €300{,}000\\

\textbf{Summary:} An open-source, offline transit routing system prioritizing privacy and cross-platform availability.\\


\section*{Form VV-A2}

\textbf{01.Name of the organization:} Slovak Technical University – FIIT\\
\textbf{02.Abbreviation:} STU FIIT\\
\textbf{03.Organization address:} Ilkovičova 2, 842 16 Bratislava 4\\
\textbf{07.Phone, email:} +421 2 210 22 143, info@fiit.stuba.sk\\

\section*{Form VV-A3}

\begin{table}[H]
\centering
\renewcommand{\arraystretch}{1.1}

\begin{tabularx}{\textwidth}{|c|X|}
\hline
\textbf{VV -- A3} & \textbf{List of participants} \\
\hline
01 & \textbf{List of staff directly involved in project} \\
\hline
\end{tabularx}

\vspace{0.3cm}

\begin{tabular}{|p{3.5cm}|p{2.5cm}|p{2.0cm}|p{2.5cm}|p{2cm}|p{2cm}|}
\hline
\textbf{Name and surname} & \textbf{Titles} & \textbf{Position} & 
\textbf{Date of birth} & \textbf{Hours} & \textbf{Hours in 2 yrs} \\
\hline
Yehor Vinnyk & None & Developer & 04.12.2007 & 6 & 2880 \\ \hline
Dao Anh Khoi Vuong & None & Developer & 29.08.2007 & 6 & 2880 \\ \hline
Lukáš Mlynarčík & None & Designer & 20.02.2006 & 6 & 2880 \\ \hline
\end{tabular}

\end{table}

\begin{table}[H]
\centering
\renewcommand{\arraystretch}{1.3}
\begin{tabularx}{0.95\textwidth}{|c|X|X|}
\hline
\textbf{VV – A3} & \multicolumn{2}{c|}{\textbf{List of participants}} \\ \hline

\textbf{02} & \textbf{Total number of other staff} & 12 \\ \hline
            & \textbf{Total capacity of other staff in hours} & 42{,}240 \\ \hline

\textbf{03} & \textbf{Total number of involved staff} & 15 \\ \hline
            & \textbf{Total capacity of involved staff in hours} & 50{,}880 \\ \hline
\end{tabularx}
\end{table}


\section*{Form VV-A4}

\renewcommand{\arraystretch}{1.25}

\begin{longtable}{|p{0.10\textwidth}|p{0.40\textwidth}|p{0.40\textwidth}|}
\hline
\multicolumn{3}{|c|}{%
\parbox{0.9\textwidth}{\centering\textbf{ Basic information on the Principal Investigator}}
} \\ \hline

\textbf{01} & \textbf{Name and surname} & Jan Lang \\ \hline
\textbf{02} & \textbf{Gender} & Male \\ \hline
\textbf{04} & \textbf{Email} & jan.lang[at]stuba.sk \\ \hline




\textbf{05} & \textbf{Principal investigator is young investigator} & No \\ \hline




\textbf{07} & \textbf{List of PI projects} &Modelom riadená identifikácia vzdelávacieho obsahu s podporou kolaboratívnej tvorby otázok a úloh(supervisor), Modelom riadená identifikácia vzdelávacieho obsahu s podporou kolaboratívnej tvorby otázok a úloh(supervisor), Modelom riadená identifikácia vzdelávacieho obsahu s podporou kolaboratívnej tvorby otázok a úloh(sipervisor),  Model driven identification of the educational content with coalborative questions and answers creation support (MoD2CIE)(supervisor), 	 Model driven identification of the educational content with coalborative questions and answers creation support(supervisor) \\ \hline

\textbf{07} & \textbf{Number of PI projects} & 5 \\ \hline



\end{longtable}


\section*{Form VV-B: Project Objectives and Outputs}
\textbf{Keywords:} Open-source, Public Transit, Offline Routing, Privacy, Mobile Platforms\\ 
\textbf{General Objective:} Develop a cross-platform, open-source transit route planner emphasizing privacy and offline functionality.\\
\textbf{Specific Objectives:}
\begin{itemize}
  \item Implement detailed route customization.
  \item Ensure offline operation.
  \item Provide transparent data handling.
  \item Support extended platforms.
\end{itemize}
\textbf{Project's TRL level:} 4\\ 
\textbf{Explanation of TRL level:} The project is currently at Technology Readiness Level 4 (TRL 4). The core concept, system architecture, and key functionalities (offline routing, privacy-preserving data handling, downloadable timetables, and community-based data updates) have been designed and validated at a conceptual level. Initial prototypes of individual components have been implemented and tested in a controlled development environment; however, the complete integrated system has not yet been demonstrated in an operational or real-world setting.\\ 
\textbf{Inclusion of project outputs in the relevant TRL scale with description of outputs:} The project outputs will move the technology from TRL 4 to TRL 6.
Expected outputs include:

A fully functional offline-capable public transport planning application with downloadable map and timetable databases.

An integrated routing engine supporting user-defined preferences (transport operators, transfers, walking distance).

A privacy-preserving data model without user tracking or mandatory online connectivity.

A community-based data editing and validation system.

Demonstration of the integrated solution in a relevant real-world environment with real public transport data.\\ 
\textbf{Project outcomes applications in practice – Outcomes customer (user) is applicant:} The applicant will use the project outcomes for research, development, and public dissemination of an open-source, offline-capable public transport planning solution. The results will be applied to validate privacy-focused transport planning models and to support further academic research, educational activities, and open-data initiatives.\\ 
\textbf{Project outcomes applications in practice – Outcomes customer (user) is other user:} Other users include general public transport passengers, municipalities, transport authorities, public transport operators, open-data communities, and non-profit organizations interested in privacy-conscious digital mobility tools.\\ 
\textbf{An overview of the planned outputs and contributions of the project which will be publicly available:} The project will deliver a fully open-source, privacy-conscious public transport planning platform, including source code, technical documentation, datasets (where legally permissible), and user documentation. Publicly available outputs will include a cross-platform mobile and desktop application, open APIs, and tools for community-based data contributions. The project will contribute to increased digital accessibility, data transparency, and user privacy in public transport systems while demonstrating a sustainable, community-driven alternative to commercial mobility applications.\\ 

\section*{Form VV-C: Project Budget (Summary)}
\begin{longtable}{p{5cm}p{4cm}p{4cm}}
\toprule
\textbf{Category} & \textbf{Year 1} & \textbf{Year 2} \\
\midrule
Personnel Costs & 90.000 & 105.000 \\ 
Software/Hardware & 10.000 & 6.000 \\
Travel & 6.000 & 9.000 \\
Insurance & 15.000 & 15.000 \\
Energy, water, comunications & 2.500 & 2.500 \\
Popularization costs & 1.000 & 5.000 \\
Other Expenses & 4.000 & 5.000 \\
\textbf{Total} & 128.500 & 147.500 \\
\bottomrule
\end{longtable}




\section*{Form VV-D: Project Schedule and Milestones}
\textbf{Duration:} 24 months\\
\textbf{Milestones:}
\begin{itemize}
  \item M1 – Prototype engine.(7 months)
  \item M2 – Cross-platform client.(8 months)
  \item M3 – Audit.(4 months)
  \item M4 – Public release.(months)
\end{itemize}

\section*{Form VV-E: Declarations}
I declare that the submitted project proposal is original.\\[1em]

\textbf{Signature:}\\
\begin{figure}[h]
\centering
\includegraphics[width=0.4\linewidth]{Signature.png}
\end{figure}

\textbf{Date:} 15/12/2025

\end{document}
