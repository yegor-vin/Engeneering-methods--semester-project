\documentclass[12pt,a4paper]{article}
\usepackage[utf8]{inputenc}
\usepackage{geometry}
\geometry{margin=2.5cm}
\usepackage{graphicx}
\usepackage{hyperref}
\usepackage{longtable}
\usepackage{booktabs}
\usepackage{caption}
\usepackage{array} 
\usepackage{url}
\usepackage{amsmath}
\usepackage{multirow}

\title{\textbf{Decentralizing Digital Mobility: An Overview of Open-Source, Privacy-First Transit Routing Solutions}}
\author{Student Team, Supervised by Prof. Ján Lang \\ \small Intelligent Data and Systems with Applications (IDSA), FIIT STU}
\date{December 2025}

\begin{document}
\maketitle

\begin{abstract}
The ubiquitous integration of commercial platforms into public life, particularly in digital mobility, has intensified concerns regarding user data privacy, pervasive surveillance, and algorithmic opacity~\cite{google_privacy}. This article presents the technical and ethical framework for the VV-F project, which develops a fully open-source, privacy-first public transit route planner with native offline capabilities. The project's innovation lies in its complete reliance on public data standards, namely GTFS~\cite{gtfs_spec} and OpenStreetMap (OSM)~\cite{osm_wiki}, combined with a transparent, multi-criteria routing engine~\cite{delling_transit_routing}. By adhering strictly to Free and Open Source Software (FOSS) principles~\cite{opensource_values}, the initiative serves as a practical demonstrator of ethical technology and significantly contributes to both academic knowledge and social innovation by offering a reproducible, low-resource alternative to centralized proprietary services~\cite{nlnet_case_study}. The final output is an end-to-end open mobility solution, supported by a structured development methodology.
\end{abstract}

\newpage
\section{Introduction}

\subsection{The Crisis of Proprietary Mobility Services}
The digital mobility landscape is currently dominated by proprietary solutions whose business models are often predicated on extensive user tracking and data monetization~\cite{google_privacy}. This reliance on centralized infrastructure creates a dependency that compromises user privacy and leads to a lack of transparency in algorithmic decision-making. The societal need is clear: the development of genuinely ethical, transparent, and trustworthy digital mobility tools that operate independently of commercial tracking mechanisms~\cite{opensource_values}.

\subsection{The VV-F Project's Value Proposition}
The VV-F project directly addresses this need by designing and implementing a system that is fundamentally privacy-first and open-source. The originality of the project lies in a unique combination of technical objectives:
\begin{itemize}
    \item \textbf{Decentralized Operation:} Complete offline routing capability, ensuring functionality without constant server communication or data logging.
    \item \textbf{Open Data Foundation:} Exclusive use of global open datasets (GTFS and OSM).
    \item \textbf{Algorithmic Transparency:} Implementation of a multi-criteria routing algorithm that is fully visible and verifiable.
    \item \textbf{Cross-Platform Accessibility:} Deployment across major mobile platforms (Android, iOS, and mobile Linux).
\end{itemize}

The expected outcome is a functional, deployable prototype that contributes to intelligent transport systems, privacy-preserving computation, and the broader open-source mobility ecosystem.

\section{Technological Foundations and Open Data Challenges}
The project's success is predicated on the successful acquisition, validation, and processing of two mature global open data standards.

\subsection{Data Standards and Completeness}
\begin{itemize}
    \item \textbf{GTFS (General Transit Feed Specification):} This provides the authoritative data layer for all scheduled public transport services, defining routes, stops, and trips~\cite{gtfs_spec}. The challenge lies in converting the relational GTFS data into a graph structure optimized for fast, multi-criteria traversal in a resource-constrained mobile environment.
    \item \textbf{OpenStreetMap (OSM):} This is utilized to provide the crucial base map, walking networks, and geographical context~\cite{osm_wiki}. While comprehensive, the fitness of OSM data for routing applications often depends on regional completeness and quality~\cite{osm_completeness}. The methodology must incorporate validation steps to ensure that the required topological fidelity for walking segments is met in the target deployment area.
\end{itemize}

\subsection{Algorithmic Approach}
The core routing engine builds upon efficient algorithms derived from academic research. Specifically, it draws on methodologies like the station-graph model and techniques for multi-criteria path finding (such as Round-Based Public Transit Routing)~\cite{delling_transit_routing}. This approach is necessary to handle the complexity of public transport networks, which involve both scheduled events (time-dependent edge weights) and static paths (walking segments). The engine will support optimization for user-defined criteria, such as minimizing total travel time, minimizing transfers, or minimizing walking distance.

% Placeholder for Diagram 1 (New)
\begin{figure}[ht]
    \centering
    %     \includegraphics[width=0.9\textwidth]{RoutingFlowchart.pdf}
    \caption{The VV-F Routing Process Flow. Data from GTFS and OSM is pre-processed into a graph stored locally on the device. User queries are processed offline using the multi-criteria engine.}
    \label{fig:routing_flow}
\end{figure}

\section{Methodology and Ethical Development Framework}
The project follows an iterative, incremental development methodology guided by an ethical technology framework.

\subsection{FOSS as an Ethical Imperative}
The choice of FOSS is not merely technical, but an ethical commitment, reinforcing core open collaboration values~\cite{opensource_values}. This approach naturally satisfies regulatory expectations for data protection and transparency. It offers an alternative that is compliant with principles found in frameworks such as the GDPR by avoiding the collection of personal data entirely~\cite{gdpr_framework}. The FOSS model allows the project to be fully auditable, serving as a social innovation that promotes public awareness of ethical tool design.

\subsection{System Architecture and Data Flow}
The system utilizes a modular, three-layer architecture designed for offline mobile performance. This structure separates data handling, logic, and presentation. The process begins with external data ingestion and transformation into an optimized graph structure, which is then stored locally on the device.

% Placeholder for Diagram 2 (Existing in original file)
\begin{figure}[ht]
    \centering
    \includegraphics[width=0.8\textwidth]{SystemArchitecture.pdf}
    \caption{Conceptual Three-Layer Architecture for the VV-F Mobile Routing Application. The data layer uses open standards and is stored locally; the core engine performs multi-criteria routing; and the presentation layer provides a native cross-platform user interface.}
    \label{fig:architecture}
\end{figure}

The routing process then uses the locally stored graph to compute paths. This focus on local processing minimizes latency and eliminates the possibility of server-side user tracking.

\subsection{Evaluation and Validation Strategy}
The validation strategy is rigorous and two-fold:
\begin{enumerate}
    \item \textbf{Data Integrity Check:} Scripts will verify the successful transformation of raw GTFS/OSM data into the internal graph model, ensuring no routes or stops are lost or corrupted.
    \item \textbf{Comparative Performance Test:} The final prototype's output will be compared against commercial routing applications (e.g., Google Maps) and official local transit timetables. The objective is to achieve at least 80\% correctness in route calculation and travel time estimates, demonstrating technical viability.
\end{enumerate}

\section{Impact and FOSS Ecosystem Contribution}
The project’s impact is broad, extending beyond a mere technological artifact to influence social, economic, and educational spheres.

\subsection{Societal and Economic Benefits}
The VV-F solution validates that public transport routing can be provided as an infrastructure service rather than a commercial product, delivering substantial societal benefits:
\begin{itemize}
    \item \textbf{Privacy Enhancement:} Directly addresses the dependency on services reliant on user profiling, advancing ethical technology and social innovation~\cite{gdpr_framework}.
    \item \textbf{Resource Efficiency:} Offline functionality dramatically reduces mobile data consumption for users and minimizes operational server costs for potential adopters, making the solution economically sustainable and contributing to lower resource use~\cite{opensource_budget}.
    \item \textbf{Sustainable Mobility:} By providing reliable, free, and transparent routing, the project lowers the barrier to using public transport, which aligns with goals for reduced emissions and improved urban quality of life.
\end{itemize}

\subsection{Educational and Research Impact}
The development process itself is a central impact vector. As a student-led initiative, the project transforms open-source development into a formalized educational tool~\cite{opensource_learning}. It strengthens competencies in data engineering, algorithmic design, and collaborative open-source workflows. The project contributes to the applied research domain by releasing a fully reproducible mobility pipeline, which can serve as a baseline for future work in real-time multi-modal transport research. This direct link between FOSS and innovation is well-documented in the literature~\cite{foss_impact}.

\begin{table}[ht]
    \centering
    \caption{Comparative Features: Proprietary vs. VV-F Open-Source Routing}
    \label{tab:comparison}
    \begin{tabular}{l >{\centering\arraybackslash}m{4cm} >{\centering\arraybackslash}m{4cm}}
    \toprule
    \textbf{Feature} & \textbf{Proprietary Systems (e.g., Google)} & \textbf{VV-F Open-Source Project} \\
    \midrule
    Data Source & Commercial/Proprietary & GTFS, OpenStreetMap (Open) \\
    User Tracking & Extensive (Profiling) & None (Fully Offline) \\
    Transparency & Low (Black Box) & High (Open Source)~\cite{opensource_values} \\
    Resource Use & High (Constant Data) & Low (Offline)~\cite{opensource_budget} \\
    Dissemination & Closed Licensing & Open License~\cite{opensource_dissemination} \\
    \bottomrule
    \end{tabular}
\end{table}

\section{Implementation Strategy}

\subsection{Work Plan Structure}
The project is managed through four sequential work packages over 24 months, ensuring that foundational steps (data preparation) precede core development (routing engine) and final deployment (UI/evaluation).

\begin{longtable}{p{1.4cm} p{4.3cm} p{7.3cm} p{2.2cm}}
\toprule
\textbf{WP} & \textbf{Title} & \textbf{Tasks and Deliverables} & \textbf{Duration} \\
\midrule
WP1 & Research \& Data Prep. & Study GTFS and OSM standards; gather datasets; prepare import/validation scripts for the target region. & Months 1--6 \\
WP2 & Core Routing Prototype & Implement the multi-criteria routing algorithm; conduct unit testing; verify core algorithmic correctness~\cite{delling_transit_routing}. & Months 7--14 \\
WP3 & UI and Offline Deployment & Develop cross-platform UI using a FOSS framework; implement persistent offline storage and retrieval for routing graph. & Months 15--22 \\
WP4 & Evaluation and Dissemination & Compare prototype performance with commercial systems; final documentation; publication of source code, guides, and academic results~\cite{opensource_dissemination}. & Months 23--24 \\
\bottomrule
\end{longtable}

\subsection{Risk Mitigation and Supervision}
Key risks include the lack of prior student experience and the technical complexity of offline routing. These are mitigated by applying rigorous iterative testing and by leveraging academic mentorship. The project is specifically overseen by \textbf{Professor Ján Lang} from the IDSA research group at FIIT STU. His expertise in secure software engineering and intelligent systems provides essential guidance, ensuring the project maintains a high level of technical rigor and adheres to academic best practices.

\section{Dissemination and Sustainability}

\subsection{Open Publication}
All project outputs—source code, technical documentation, and derived datasets—will be published on GitHub under a permissive FOSS license. This commitment to open publication is vital for maximizing the project's impact, aligning with best practices for open research dissemination~\cite{opensource_dissemination}. By making the entire pipeline reproducible, the project lowers the barrier for other researchers or civic groups globally to adapt the tool to their local GTFS/OSM data.

\subsection{Sustainability and Future Work}
The high cost efficiency inherent in the reliance on FOSS tools and minimal operational hosting costs~\cite{opensource_budget} ensures the project's long-term sustainability. The foundation laid by the VV-F prototype opens several avenues for future research, including the integration of real-time transit data, advanced multi-modal routing (combining biking/car sharing), and more complex machine learning models for predicting transit delays, all while maintaining the core commitment to user privacy and open governance.

\section{Conclusion}
The VV-F project serves as a compelling case study for applying open-source principles to solve critical issues in modern digital infrastructure. It addresses the privacy deficit in mobility services by developing a robust, transparent, and offline-capable transit routing solution based entirely on public data standards. Through its rigorous methodology, ethical FOSS commitment, and strong academic supervision, the project promises not only a technically viable product but also a significant contribution to educational innovation and the global effort to create more ethical and user-centric technology environments. The complete open publication of the source code and documentation ensures the work's longevity and maximum utility for the international FOSS and academic communities.

\newpage
\bibliographystyle{plain}
\bibliography{References}

\end{document}